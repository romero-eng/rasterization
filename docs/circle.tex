\documentclass{article}
\usepackage{graphicx}
\usepackage{amsmath}
\DeclareMathOperator{\floor}{floor}
\begin{document}

\begin{figure}[ht!]
\centering
\includegraphics[width=90mm]{Circle Rasterization.jpg}
\caption{Circle Rasterization Algorithm \label{overflow}}
\end{figure}



\underline{\textbf{First Octant Rasterization Algorithm:}} \\
\begin{enumerate}
    \item \underline{Set Initial Values:} 
          \begin{align*}
                x[0] = r, \quad y[0] = 0 
            \end{align*}
    \item \underline{for $n = 0 \rightarrow N - 2$:} \\
          \begin{align*}
            &y[n + 1] = y[n] + 1 \\ \\
            &x[n + 1] = \begin{cases}
                            x[n] \quad \ \ \ , \quad x[n] - x\Big(y[n + 1]\Big) < 0.5 \\ \\
                            x[n] - 1 \ , \quad x[n] - x\Big(y[n + 1]\Big) \geq 0.5
                          \end{cases}
      \end{align*}
\end{enumerate}

\underline{\textbf{Initial Mathematics:}}
\begin{align*}
    &\quad \ \ x^{2} + y^{2} = r^{2} \\ \\
    &\Rightarrow x^{2} = r^{2} - y^{2} \\ \\
    &\Rightarrow x(y) = \pm\sqrt{r^{2} - y^{2}} \\ \\
    &\quad \quad \quad \ \ = \begin{cases}
                                \ \ \sqrt{r^{2} - y^{2}}, \quad y \geq 0 \quad \ \ \big(\text{Top Half-Circle}\big) \\ \\
                                -\sqrt{r^{2} - y^{2}}, \quad y < 0 \quad \big(\text{Bottom Half-Circle}\big) \\ \\ 
                               \end{cases} \\ \\
    &\quad \quad \quad \ \ = \sqrt{r^{2} - y^{2}} \quad \big(\text{Only need Top Half-Circle for Rasterization}\big) \\ \\
    &\Rightarrow \frac{dx(y)}{dy} = \frac{-y}{\sqrt{r^{2} - y^{2}}} \\ \\
\end{align*}

\underline{\textbf{Incremental Axis Calculation:}}
\begin{align*}
    &\quad \ \ y[n + 1] = y[n] + 1 \\ \\
    &\Rightarrow y[n] = y[n - 1] + 1 \\ \\
    &\quad \quad \quad \ = y[n - 2] + 2 \\ \\
    &\quad \quad \quad \ = y[n - n] + n \\ \\
    &\quad \quad \quad \ = n
\end{align*}

\underline{\textbf{Orthogonal Axis Calculation:}}
\begin{align*}
    x\Big(y[n + 1]\Big) &= \sqrt{r^{2} - y^{2}[n + 1]} \\ \\
                        &= \sqrt{r^{2} - \big(n + 1\big)^{2}} \\ \\
\end{align*}

\underline{\textbf{Decision Optimization:}}
\begin{align*}
    &\quad \quad \quad \ \ x[n] - x\Big(y[n + 1]\Big) \geq 0.5 \\ \\
    &\quad \quad \Rightarrow 2x[n] - 2x\Big(y[n + 1]\Big) \geq 1 \quad \big(\text{Integer Bitshift Modification}\big) \\ \\
    &\quad \quad \Rightarrow 2x[n] - 2\sqrt{r^{2} - \big(n + 1\big)^{2}} \geq 1 \\ \\
    &\quad \quad \Rightarrow 2\sqrt{r^{2} - \big(n + 1\big)^{2}} \leq 2x[n] - 1 \\ \\
    &\quad \quad \Rightarrow 4\Big(r^{2} - \big(n + 1\big)^{2}\Big) \leq \Big(2x[n] - 1\Big)^{2} \quad \big(\text{Only possible since we remain in first quadrant}\big) \\ \\
    &\quad \quad \Rightarrow 4\Big(x^{2}[n] - x[n] + n^{2} + 2n\Big) \geq \tau, \quad \tau \equiv 4r^{2} - 5 \\ \\
\end{align*}

\underline{\textbf{Number of elements in first octant:}}
\begin{align*}
    &\quad \quad \frac{dx\Big(y[N - 1]\Big)}{dy} \leq -1 \quad \big(N - 1 \text{ instead of just } N \text{ due to zero-based indexing}\big) \\ \\
    &\Rightarrow \frac{-y[N - 1]}{\sqrt{r^{2} - y^{2}[N - 1]}} \leq -1 \\ \\
    &\Rightarrow \frac{-(N - 1)}{\sqrt{r^{2} - \big(N - 1\big)^{2}}} \leq -1 \\ \\
    &\Rightarrow N - 1 \leq \frac{r}{\sqrt{2}} \\ \\
    &\Rightarrow N = \floor\Bigg(\frac{r}{\sqrt{2}}\Bigg) + 1\\ \\
\end{align*}

\big(\textbf{Caution}\big) When $y[N - 1] = x[N - 1]$:
\begin{align*}
    & \arctan\Bigg(\frac{r}{\sqrt{2}}\Bigg) = \arctan\Bigg(\frac{y[N - 1]}{x[N - 1]}\Bigg) = 45^{\circ} \\
\end{align*}

Typically, the array sizes of the first and second octants are usually the same, but when $y[N - 1] = x[N - 1]$,
there exists a center point in between the first and second octant which this algorithm counts as part of the first
octant since it technically fulfills the boolean condition up above (e.g., r = 20). It must be checked and accounted for when filling
in the second octant. \\ \\

\underline{\textbf{Final Rasterization Algorithm:}} \\
\begin{enumerate}
    \item \underline{Set Initial Values:} 
          \begin{align*}
                &x[0] = r, \quad y[0] = 0 \\ \\
                &N = \floor\Bigg(\frac{r}{\sqrt{2}}\Bigg) \\ \\
                &\tau = 4r^{2} - 5
            \end{align*}
    \item \underline{for $n = 0 \rightarrow N - 2$:} \\
          \begin{align*}
            &y[n + 1] = y[n] + 1 \\ \\
            &x[n + 1] = x[n] - d, \quad d \equiv \begin{cases}
                                                     0 \ , \quad 4\Big(x^{2}[n] - x[n] + n^{2} + 2n\Big) < \tau \\ \\
                                                     1 \ , \quad 4\Big(x^{2}[n] - x[n] + n^{2} + 2n\Big) \geq \tau
                                                   \end{cases}
          \end{align*}
    \item \underline{Fill in other octants/quadrants} \\ \\

          \begin{align*}
                M \equiv N - p, \quad p \equiv \begin{cases}
                                                   1, \quad y[N - 1] = x[N - 1] \\ \\
                                                   0, \quad y[N - 1] < x[N - 1]
                                                 \end{cases} \\ \\
                Q \equiv N + M \\ \\
                T \equiv 2*Q - 1 \\ \\
                C \equiv 2*(T - 1) \\ \\
            \end{align*} 

          \underline{for $n = 0 \rightarrow N - 1$:} \\
            \begin{align*}
                &x_{f}[n] = x[n] \\ \\
                &y_{f}[n] = y[n] \\ \\
              \end{align*}
          \underline{for $m = N \rightarrow m = Q - 1$:} \\
            \begin{align*}
                &x_{f}[m] = y[M - 1 - m] \\ \\
                &y_{f}[m] = x[M - 1 - m] \\ \\
              \end{align*}
          \underline{for $q = Q \rightarrow q = T - 1$:} \\
             \begin{align*}
                &x_{f}[q] =    -x_{f}[T - 1 - q] \\ \\
                &y_{f}[q] = \ \ y_{f}[T - 1 - q] \\ \\
               \end{align*}
          \underline{for $t = T \rightarrow t = C - 1$:} \\
             \begin{align*}
                &x_{f}[t] = \ \ x_{f}[C - t] \\ \\
                &y_{f}[t] =    -y_{f}[C - t] \\ \\
               \end{align*}
\end{enumerate}

\end{document}
